\documentclass[12pt]{article}
\usepackage[utf8]{inputenc}
\usepackage[T1]{fontenc}
\usepackage{amsmath,amsfonts,amssymb}
\usepackage{graphicx}
\usepackage{a4wide}
\title{Reconstructed abstract of the paper 

``Comprehensive study of feature selection methods to solve multicollinearity problem according to evaluation criteria''}
\date{}
\begin{document}
\maketitle

\begin{abstract}
This paper presents a detailed study of addressing the multicollinearity problem in data fitting through a quadratic programming approach for feature selection. The proposed method aims to reduce model redundancy and instability caused by feature collinearity. By defining a quadratic objective function based on feature relevance and similarity measures, the method selects features that provide better stability and reduced redundancy. Compared with other feature selection methods, such as LARS, Lasso, and Genetic algorithms, the proposed approach demonstrates improved performance in experiments on both synthetic and real datasets.
\end{abstract}

\paragraph{Keywords:} multicollinearity, feature selection, quadratic programming, data fitting, evaluation criteria, correlation coefficient, mutual information

\paragraph{Highlights:}
\begin{enumerate}
    \item The study addresses the multicollinearity problem using a quadratic programming approach.
    \item It reduces redundancy and instability in data models by selecting relevant features.
    \item Compared to LARS, Lasso, and Ridge methods, this approach outperforms others in terms of stability and quality of the selected features.
\end{enumerate}

\end{document}
