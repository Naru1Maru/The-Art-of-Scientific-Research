\begin{frame}{Прогнозирование оттока клиентов}
Машинное обучение для точного прогнозирования оттока клиентов и реализации проактивных стратегий удержания.

\begin{block}{Проблема}
Как определить клиентов с высокой вероятностью ухода?
\end{block}

\begin{block}{Предложенный метод}
Разработка модели машинного обучения, анализирующей данные о клиентах: демографические характеристики, историю транзакций, паттерны использования и взаимодействие с поддержкой.
\end{block}

\begin{block}{Решение}
Полученные результаты позволят компании снизить уровень оттока клиентов:
\begin{enumerate}[1)]
    \item Настроить предсказательную модель,
    \item Проанализировать данные о клиентах,
    \item Реализовать проактивные стратегии удержания.
\end{enumerate}
\end{block}
\end{frame}

\begin{frame}{Визуальные акценты как форма передачи сообщения%
\footnote{\textit{В.\,В.~Стрижов и др.} Аналитические и стохастические методы оценки параметров структуры~// Информатика, 2016.}}
Что аудитория видит на графике:
\begin{columns}
\begin{column}{0.3\textwidth}
\begin{enumerate}[1)]
    \item Характеристики клиентов (возраст, частота использования),
    \item Влияние характеристик на вероятность оттока.
\end{enumerate}
\end{column}
\begin{column}{0.7\textwidth}
    \includegraphics[width=1\textwidth]{Name-Step-3-fig}      
\end{column}
\end{columns}
\bigskip
Каковы последствия? Более яркие цвета соответствуют повышенному риску, что позволяет фокусироваться на клиентах с высокой вероятностью ухода.
\end{frame}
